%% Sample file showing how to use the listings files.

\documentclass[11pt]{article}     
\usepackage{graphicx}
\usepackage{listings}
\usepackage{listings-sweave-xelatex}
\usepackage[svgnames]{xcolor}
\usepackage{soul}
\sethlcolor{LightGoldenrodYellow}

\usepackage[dvipdfm,colorlinks=true,urlcolor=FireBrick,linkcolor=FireBrick,bookmarks=false,citecolor=Black]{hyperref}

\begin{document}
%\frontmatter
\title{R/Sweave Code Highlighting Examples}

\author{Kieran Healy}
\date{{\scriptsize{\texttt{\href{mailto:kjhealy@soc.duke.edu}{kjhealy@soc.duke.edu}}}}} 

\maketitle

The files in this folder describe an Sweave style for the Listings package. The nesting in the \texttt{.tex} file might looks a little confusing at first because of the way we have to switch back and forth between Sweave and LaTeX. But if you're just typesetting R it's much easier. See the documentation for the Listings package for much, much more detail.

\section*{Some R}

\begin{figure}[h!] % h! forces the figure to appear right here. 
% This just does the box. There's certainly a more elegant way to do this.
% \begin{lstlisting}[style=sweave-top]

% \end{lstlisting} 
\begin{lstlisting}[language=R,numbers=none,style=rcodebox] 

turkey.fun <- function(x){
 out<- log(x)*1.65
 out
 }

x.ax <- seq(10,35,by=0.01)
hours <- turkey.fun(x.ax)
data<-cbind(x.ax,hours)
colnames(data) <- c("Weight","Hours")
data <- data.frame(data) 
\end{lstlisting}
% \begin{lstlisting}[style=sweave-bottom]

% \end{lstlisting} 
 \caption{A chunk of R code.}
\label{fig:rchunk}
\end{figure}

Here is some R as rendered by the Listings package and wrapped in a Figure environment. Change the style options by editing the \texttt{rcodebox} style definition in  \texttt{listings-sweave-xelatex.sty} file or defining your own. The Listings manual has the details.

\section*{An Sweave Example} % (fold)
\label{sec:an_sweave_example}
Here we show the code listing for a longer Sweave example. The \texttt{lstlisting} statements flip back and forth between sweave-r and sweave-tex styles. 
 
\begin{figure}[h!]
\begin{lstlisting}[style=sweave-top]

\end{lstlisting} 
\begin{lstlisting}[language={[latex]tex},numbers=none,style=sweave-tex]   
\subsection{Some exploratory analysis}
\label{sec:exploratory}
In this section we do some exploratory analysis of the data. 
We begin by reading in the data file:
\end{lstlisting}
\begin{lstlisting}[language=R,numbers=none,style=sweave-r] 
<<load-data, echo=true>>=
# load the data. 
my.data <- read.csv("data/sampledata.csv",header=TRUE)

# OLS model
out <- lm(y ~ x1 + x2,data=my.data)

summary(out)

# ... More R code would follow until the end delimiter:

@ 
\end{lstlisting}
\begin{lstlisting}[language={[latex]tex},numbers=none,style=sweave-tex] 
% now we are back to normal latex 
This concludes the exploratory analysis. 
\end{lstlisting} 
\begin{lstlisting}[style=sweave-bottom]

\end{lstlisting}
  \caption{A chunk of Sweave code in a LaTeX document.}
\label{fig:codechunk}
\end{figure}

One thing to notice in source file for the next example (below) is that we use an escaped character to activate a command, defined in \texttt{listings-sweave.sty}, which allows you to highlight Sweave or R expressions inline. 

\begin{figure}
\begin{lstlisting}[style=sweave-top]

\end{lstlisting} 
\begin{lstlisting}[language={[latex]tex},numbers=none,style=sweave-tex]   
\subsection{Appendix: An Sweave Example}
\label{sec:append-sweave-exampl}
\end{lstlisting}
\begin{lstlisting}[language=R,numbers=none,style=sweave-r] 
<<prelim, echo=false,results=hide>>=
# load required libraries and set some options.
options(device="pdf")
library(lattice)
library(xtable)
data(cats, package="MASS")
@ 
\end{lstlisting}

%% Note the use of escapechar and the \HL / \HLoff commands (defined in
%% listings-sweave.sty), to allow me to highlight \Sexpr expressions inline. 
\begin{lstlisting}[language={[latex]tex},numbers=none,style=sweave-tex] 
Consider the \texttt{cats} regression example. The data contains
measurements of heart and body weight of
#\HL#\Sexpr{nrow(cats)}#\HLoff# cats (\Sexpr{sum(cats$Sex=="F")}
female, #\HL#\Sexpr{sum(cats$Sex=="M")}#\HLoff# male). A linear
regression model of heart weight by bodyweight and sex can be fitted
in R using the command
\end{lstlisting} 
\begin{lstlisting}[language=R,numbers=none,style=sweave-r]
<<ols.model>>= 
# OLS regression model 
lm1 <- lm(Hwt~Bwt*Sex,data=cats) 
@
\end{lstlisting}
\begin{lstlisting}[language={[latex]tex},numbers=none,style=sweave-tex]
Tests for significance of the coefficients are shown in
Table~\ref{tab:coef}, a scatter plot including the regression lines 
is shown in Figure~\ref{fig:cats}.
\end{lstlisting}

\begin{lstlisting}[language=R,numbers=none,style=sweave-r]
<<summary.table,results=tex,echo=FALSE>>= 
# make a table summarizing the results 
xtable(lm1, caption="Linear regression model for cats
  data.", label="tab:coef") 
@
\end{lstlisting}
\begin{lstlisting}[language={[latex]tex},numbers=none,style=sweave-tex]
In this example, the code chunks that produce Table \ref{tab:coef} 
and Figure \ref{fig:cats} are replaced by their output when the 
final document is produced.

\begin{figure} \centering
\end{lstlisting}
\begin{lstlisting}[language=R,numbers=none,style=sweave-r]
<<ols.figure,fig=TRUE,echo=FALSE>>= 
# produce a plot of the regression
# lines by sex 
print(xyplot(Hwt~Bwt|Sex, data=cats,type=c("p","r"))) 
@
\end{lstlisting}
\begin{lstlisting}[language={[latex]tex},numbers=none,style=sweave-tex]
  \caption{The cats data from package MASS.}
  \label{fig:cats}
\end{figure}
 
The code used to produce the data analysis shown here is presented in
Figure \ref{fig:code}.
\end{lstlisting}
\begin{lstlisting}[style=sweave-bottom]

\end{lstlisting}
  \caption{Some Sweave code. The text is a mixture of LaTeX markup and
chunks of R code (shown here in blocks with yellow background).}
\label{fig:code}
\end{figure}
 

\end{document}
